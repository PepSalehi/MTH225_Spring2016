\section{Frequentist and Bayesian Inference}
There are two types of statistical inference, classical or \textit{frequentist} and \textit{Bayesian}.
\par\vspace{0.5 cm}
\subsection{Frequentist Inference}
In classical or frequentist inference, 
\par\vspace{0.5 cm}
\begin{itemize}
\item Data values are considered to be random variables.
\item Parameters are assumed to be fixed, unknown constants.
\item Inference is based on hypothetical replication of the experiment that produced the data many times.
\end{itemize}
\par\vspace{0.5 cm}
\subsection{Bayesian Inference}
In Bayesian inference, 
\par\vspace{0.5 cm}
\begin{itemize}
\item Data values are considered to be known values, observed from a probability distribution with associated \textit{likelihood function}
\item Parameters are assumed to be random variables.
\item Each parameter has a \textit{prior} distribution representing the experimenter's belief about the likely values of the parameter, expressed in the form of a probability distribution.
\item Inference is based on the \textit{posterior} distribution, which is a combination of the likelihood and the prior distribution, in effect a compromise.
\end{itemize}
\par\vspace{0.5 cm}
A fully specified Bayesian model consists of:
\par\vspace{0.5 cm}
\begin{itemize}
\item A likelihood function for the data
\item A prior distribution for the parameter(s)
\item A posterior distribution, which is a combination of the prior and the likelihood
\end{itemize}
\par\vspace{0.5 cm}
The steps in Bayesian inference are:
\par\vspace{0.5 cm}
\begin{itemize}
\item Define a prior distribution for each parameter
\item Define a likelihood function for the data
\item Use software such as STAN to draw a sample from the posterior distribution
\item Use the draw from the posterior distribution to estimate quantities of interest like means, standard deviations, and percentiles.
\end{itemize}
